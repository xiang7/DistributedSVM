\documentclass[10pt,conference,compsocconf,letterpaper]{IEEEtran}
%\documentclass{article}
\title{Distributed SVM}
\author{Luojie Xiang}
\date{}

\begin{document}
\maketitle{}
\section{Introduction}

Support Vector Machines (SVM) is an important machine learning algorithm. The basic SVM is a binary classifier which seeks to find a hyperplane that separates data points into two classes. SVM is a max-margin classifier which tries to maximize the distance of the hyperplane to the boundary points of each class. The boundary points are called support vectors and thus the name support vector machine. 

SVM has many advantages over other machine learning algorithms. Its decision relies only on support vectors. This means, once trained, it make future predictions very quickly due to the few amount of support vectors. The training itself is a well-formed convex optimization problem so that convergence is guaranteed. 

The idea of the project is to design and implement a distributed SVM with the following two requirements:

\begin{itemize}
\item Faster training than standard SVM
\item Better resiliency towards malicious attack
\end{itemize}

The attack model and attack algorithms are described in next two sections.

\section{Attack Model}

This work assumes all computers can be trusted. The attack comes from the training dataset. The adversary has the following capabilities:

\begin{itemize}
\item Know the victim's dataset. 
\item Has computing power to create data points.
\item Can insert data points into the victim's dataset.
\end{itemize}

This is a practical assumption. The difference of attackers comes in the algorithm they use to create data points. Different algorithms may result in different damage power on the victim's SVM accuracy.

Next session looks into attack algorithms.


\section{Attack Algorithm}

Label flip attacks choose data points from the victim's benign training set, flip their labels and then insert them back into the victim's training set. It is a very popular attack and is well studied in literature \cite{xiao12}. 

This research focus on label flip attacks for the following reason:

\begin{itemize}
\item Label flip attack can be a very strong attack when the data points whose labels are to be flipped is chosen carefully.
\item Label flip attack requires few computational power. Adversary only needs to choose data points and flip their label. However, huge damage can be done to victim's learner by this small amount of computation. 
\item Label flip attack represents a broad family of attacks that tries to put data points of a certain class into the other class's region. Optimization attacks are one example. It requires solving an optimization over a objective function (e.g. maximize the loss function). It will result in a larger damage but require much more computation power and could be easily filtered out since they're normally very far away from the benign data points.
\end{itemize}

Our work picks three label flip attacks from a previous work with different strategy in terms of how points are picked. Difference in strategy results in the attack being stronger or weaker than others.

\begin{itemize}
\item Nearest-First Flip attack - train an SVM on victim's data set and pick the points nearest to the decision plane to flip label \cite{xiao12}.
\item Random Flip attack - randomly pick data points from victim's benign training set, and flip their labels \cite{xiao12}.
\item Furthest-First Flip attack - train an SVM on victim's data set and pick the furthest data point from the decision plane and flip its label \cite{xiao12}.
\end{itemize}

\subsection{Nearest-First Flip attack}

This attack picks the nearst points to the decision plane. Despite the fact that this is the weakest attack of the three, this attack hides the malicious points very close the the benign points. This tests the defense algorithm's robustness.

The attack works as follows:

\begin{itemize}
\item Step 1 - Train an SVM on benign data set.
\item Step 2 - Apply the decision function of SVM on all data points of the benign data set.
\item Step 3 - Sort the decision values and find the minimum in absolute value.
\item Step 4 - Flip the chosen point and insert it back into the benign data set.
\end{itemize}

\subsection{Random Flip Attack}

This attack picks a random point to flip label. It is stronger the the Nearest-First Flip attack but weaker than the Furthest-First Flip attack. The pro of this attack is that there's no need to train SVM on victim's dataset. A random pick is all computation that needs to be done. It is the fastest attack of the three and yield a good damage on victim's learner.

\subsection{Furthest-First Flip Attack}

This attack picks the data point that is furthest from the SVM decision plane and flip its label. It is shown to generate a near-optimal attack effect on SVM \cite{xiao12}. Another benefit of this attack over the optimal attack is, it works on discrete feature space, since optimal attack involves solving convex optimization over loss function which requires a differentiable field. This limits optimal attacks to only continuous feature space. This attack is the strongest of the three.

The attack works as follows:

\begin{itemize}
\item Step 1 - Train an SVM on benign data set.
\item Step 2 - Apply the decision function of SVM on all data points of the benign data set.
\item Step 3 - Sort the decision values and find the maximum in absolute value.
\item Step 4 - Flip the chosen point and insert it back into the benign data set.
\end{itemize}

\section{SVM Training Algorithm}

The design of SVM training algorithm should be parallelizable so that the training algorithm could be faster and has better tolerance against malicious attacks than standard SVM algorithm.

\textbf{1. Baseline: Ensemble Learning}

Ensemble Learning is a classic idea of improving any machine learning algorithm's resilliency against noise in the training dataset \cite{leo96, dong05}. It has also been used as a defense strategy against malicious attacks \cite{marco08, gabriela08}. 

Ensemble learning is used as our baseline, in which the training happens as follows:

\begin{itemize}
\item Step 1 - Create $N$ samples out of the training set by sampling with replacement.
\item Step 2 - Scatter $N$ samples to $N$ machines.
\item Step 3 - Train an SVM on each machine.
\item Step 4 - All $N$ SVMs vote on new data points during prediction.
\end{itemize}

\textbf{2. A better Algorithm: Modified Adaptive Cluster}

Adaptive Cluster was proposed for reducing training set size for SVM \cite{boley04, hwanjo03}, based on the idea that, SVM's decision plane relies only on support vectors \cite{koggalage04}. The process is, 

\begin{itemize}
\item Step 1 - Cluster the training set.
\item Step 2 - Train an initial SVM based on the representatives of the clusters (such as centroid).
\item Step 3 - Train a final SVM using the clusters that contain support vectors in the initial SVM.
\end{itemize}

This method is modified to include resiliency against malicious attacks.

\begin{itemize}
\item Change 1 - After step 1 of original Adaptive Cluster algorithm, add a sanitization step, using either distance filtering, or active learning (explained later).
\item Change 2 - Use the Adaptive Cluster algorithm in the ensemble framework. Step 3 in ensemble learning algorithm is changed to Adaptive Cluster algorithm above.
\end{itemize}

\textit{Distance filtering} - Malicious data points will be far away from benign data points. For example, label flip attack \cite{xiao12} picks benign data, say positive class, flip the label to negative class and insert it back to the training set. These data points will reside in the positive area except now they all have negative label. Therefore, the malicious points (with negative label) will be very far away with the benign negative points. Thus, they're very likely to be put into a separate cluster if the negative points are clustered and this cluster is very likely to be very far away from the other negative clusters. Using some distance filtering, such as probability distribution threshold will identify the malicious cluster.

\textit{Active Learning} - Active learning have machine learning algorithms summarizes the dataset and present to human experts the data points that it is most not sure of \cite{reghavan06}. Human experts will label these very concise summaries and guide the learning process. Since in Adaptive Cluster algorithm, the representatives of each cluster is found, they can be convieniently used as the summary. Human expert will tell the learning algorithm which representatives are malicious. The cluster whose representative is labeled malicious by the expert will be thrown away.

\section{Evaluation}

The evaluation comes in three fold:

\textit{Speed} - The speed would be compared with standard SVM training.

Standard SVM	- 100, 1000, 10,000 12,000 ... too long to train
Distributed SVM	- 10,000 12,000 ... 

Record Acc, Time

\textit{Resiliency against malicious attacks} - The accuracy will be compared with standard SVM as more malicious points are added.

Standard SVM    - 10,000 +2,000 malicious point ... too long to train
Distributed SVM - 10,000 +2,000 ... 

Record Acc

\subsection{Dataset}
A lot of datasets are made publicly available thanks to competitions such as KDD CUP, Kaggle and research groups such as UCI Machine Learning Repository, LIBSVM \cite{kddcup,kaggle,uci,libsvm}. Appropriate dataset will be selected later to facilitate testing.

\begin{table*}[t]
\centering
\begin{tabular}{|l|l|l|l|l|}
\hline
Dataset&Source & \# of classes & \# of data (training/testing) & \# of features (training/testing) \\
\hline
w8a&\cite{jp98} & 2 & 49,749 / 14,951 & 300 / 300 \\
\hline
\end{tabular}
\end{table*}

\section{Results}




\begin{thebibliography}{9}

\bibitem{leo96}
Breiman, Leo. "Bagging predictors." Machine learning 24, no. 2 (1996): 123-140.
\bibitem{marco08}
Barreno, Marco, Peter L. Bartlett, Fuching Jack Chi, Anthony D. Joseph, Blaine Nelson, Benjamin IP Rubinstein, Udam Saini, and J. Doug Tygar. "Open problems in the security of learning." In Proceedings of the 1st ACM workshop on Workshop on AISec, pp. 19-26. ACM, 2008.
\bibitem{dong05}
Dong, Yan-Shi, and Ke-Song Han. "Boosting SVM classifiers by ensemble." In Special interest tracks and posters of the 14th international conference on World Wide Web, pp. 1072-1073. ACM, 2005.
\bibitem{gabriela08}
Cretu, Gabriela F., Angelos Stavrou, Michael E. Locasto, Salvatore J. Stolfo, and Angelos D. Keromytis. "Casting out demons: Sanitizing training data for anomaly sensors." In Security and Privacy, 2008. SP 2008. IEEE Symposium on, pp. 81-95. IEEE, 2008.
\bibitem{boley04}
Boley, Daniel, and Dongwei Cao. "Training Support Vector Machines Using Adaptive Clustering." In SDM. 2004.
\bibitem{hwanjo03}
Yu, Hwanjo, Jiong Yang, and Jiawei Han. "Classifying large data sets using SVMs with hierarchical clusters." In Proceedings of the ninth ACM SIGKDD international conference on Knowledge discovery and data mining, pp. 306-315. ACM, 2003.
\bibitem{koggalage04}
Koggalage, Ravindra, and Saman Halgamuge. "Reducing the number of training samples for fast support vector machine classification." Neural Information Processing-Letters and Reviews 2, no. 3 (2004): 57-65.
\bibitem{xiao12}
Xiao, Han, Huang Xiao, and Claudia Eckert. "Adversarial Label Flips Attack on Support Vector Machines." In ECAI, pp. 870-875. 2012.
\bibitem{reghavan06}
Raghavan, Hema, Omid Madani, and Rosie Jones. "Active learning with feedback on features and instances." The Journal of Machine Learning Research 7 (2006): 1655-1686.
\bibitem{biggio12}
Battista Biggio, Blaine Nelson, Pavel Laskov: Poisoning Attacks against Support Vector Machines. ICML 2012
\bibitem{newsome06}
Newsome, James, Brad Karp, and Dawn Song. "Paragraph: Thwarting signature learning by training maliciously." In Recent advances in intrusion detection, pp. 81-105. Springer Berlin Heidelberg, 2006.
\bibitem{kddcup}
http://www.sigkdd.org/kddcup/index.php
\bibitem{kaggle}
http://www.kaggle.com/
\bibitem{uci}
http://archive.ics.uci.edu/ml/
\bibitem{libsvm}
http://www.csie.ntu.edu.tw/~cjlin/libsvmtools/datasets/
\bibitem{jp98}
John C. Platt. "Fast training of support vector machines using sequential minimal optimization". In Bernhard Schölkopf, Christopher J. C. Burges, and Alexander J. Smola, editors, Advances in Kernel Methods - Support Vector Learning, Cambridge, MA, 1998. MIT Press.
\end{thebibliography}

\end{document}
